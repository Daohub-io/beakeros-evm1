\documentclass[english,a4paper]{article}
\usepackage{lmodern}
\usepackage{float}
\usepackage{amssymb,amsmath}
\usepackage{ifxetex,ifluatex}
\usepackage{multirow,tabularx}
\usepackage{fixltx2e} % provides \textsubscript
\usepackage[T1]{fontenc}
\usepackage[utf8]{inputenc}
% use upquote if available, for straight quotes in verbatim environments
\IfFileExists{upquote.sty}{\usepackage{upquote}}{}

\usepackage{microtype}
\usepackage[unicode=true]{hyperref}
\usepackage[usenames,dvipsnames]{color}
\hypersetup{breaklinks=true,
            bookmarks=true,
            pdfauthor={},
            pdftitle={},
            colorlinks=true,
            citecolor=blue,
            urlcolor=blue,
            %linkcolor=magenta,
            pdfborder={0 0 0}}
\urlstyle{same}  % don't use monospace font for urls
\usepackage{color}
\usepackage{listings}

\usepackage{longtable,booktabs}
\usepackage{graphicx,grffile}
\makeatletter
\def\maxwidth{\ifdim\Gin@nat@width>\linewidth\linewidth\else\Gin@nat@width\fi}
\def\maxheight{\ifdim\Gin@nat@height>\textheight\textheight\else\Gin@nat@height\fi}
\makeatother
% Scale images if necessary, so that they will not overflow the page
% margins by default, and it is still possible to overwrite the defaults
% using explicit options in \includegraphics[width, height, ...]{}
\setkeys{Gin}{width=\maxwidth,height=\maxheight,keepaspectratio}
\setlength{\parindent}{0pt}
\setlength{\parskip}{6pt plus 2pt minus 1pt}
\setlength{\emergencystretch}{3em}  % prevent overfull lines


% Redefines (sub)paragraphs to behave more like sections
\ifx\subsubsection\undefined\else
\let\oldparagraph\subsubsection
\renewcommand{\subsubsection}[1]{\oldparagraph{#1}\mbox{}}
\fi
\ifx\subparagraph\undefined\else
\let\oldsubparagraph\subparagraph
\renewcommand{\subparagraph}[1]{\oldsubparagraph{#1}\mbox{}}
\fi

{\renewcommand{\arraystretch}{1.5}

\setcounter{secnumdepth}{3}
\setcounter{tocdepth}{3}

\title{Beaker Specification}\label{beaker-specification}

\author{Jacob Payne \and Jake O'Shannessy \and Alexey Troitskiy}

\begin{document}

\maketitle

\begin{abstract}\label{abstract}

We describe a secure and extensible operating system for smart contracts. Using
a capability-based exokernel protocol, we can facilitate secure isolation,
perform upgrades in a secure and robust manner, and prevent privilege escalation
at any point in the development process. The protocol is intended to serve as an
open standard and building block upon which more advanced and tailored security
models can be built.

\end{abstract}

\newpage
\tableofcontents
\newpage

\section{Definitions}\label{definitions}

\subsection{Kernel Instance}
A kernel instance is an Ethereum contract which using the Beaker kernel. It is
this contract that holds the storage data etc. of the system.

\section{Kernel Storage}\label{kernel-storage}
This is a critical feature. \emph{All} data stored by the kernel is stored in a
region of \emph{storage} called \emph{kernel storage}. \emph{No} other data can
be stored in \emph{kernel storage}.

This kernel storage is defined as all storage locations within a certain storage
range. All values in kernel storage share the same storage key prefix. This
prefix is currently \texttt{0xff ff ff ff}, and so currently includes all values in the interval
[\texttt{0xffffffff00} $\cdots$ \texttt{00}, \texttt{0xffffffffff} $\cdots$ \texttt{ff}].

\begin{table}[H]
  \caption{Overview of kernel storage layout.}
  \centering{}%
  \begin{tabularx}{\textwidth}{c | c | c | c | c | X }
    \hline
    \texttt{0x} & \texttt{00 - 03} & \texttt{04} & \texttt{05 - 1f} & \texttt{1e - 20} & Description \\
    \hline
    \hline
    \texttt{0x} & \texttt{ff ff ff ff} & \texttt{00} & Proc Key & $\cdots$ & Procedure Heap \\
    \texttt{0x} & \texttt{ff ff ff ff} & \texttt{01} & \texttt{00 $\cdots$ 00} & \texttt{00 00 00} & \# of Procedures\\
    \texttt{0x} & \texttt{ff ff ff ff} & \texttt{01} & Key Index & \texttt{00 00 00} & Procedure List \\
    \texttt{0x} & \texttt{ff ff ff ff} & \texttt{02} & \texttt{00 $\cdots$ 00} & \texttt{00 00 00} & Kernel Address \\
    \texttt{0x} & \texttt{ff ff ff ff} & \texttt{03} & \texttt{00 $\cdots$ 00} & \texttt{00 00 00} & Current Procedure \\
    \texttt{0x} & \texttt{ff ff ff ff} & \texttt{04} & \texttt{00 $\cdots$ 00} & \texttt{00 00 00} & Entry Procedure \\
    \hline
  \end{tabularx}
\end{table}

\section{Procedures}\label{procedures}
Data about procedure is store in 2 locations: the procedure table, and the
procedure heap. The procedure table is simply a managed, enumerable, and
iterable list of procedure keys (procedure key being something that identifies a
particular procedure). The data associated with a procedure, such as its
Ethereum address and its capabilities are stored on the procedure heap. Given a
procedure key, the data associated with a procedure can be located using that
key.

\subsection{Procedure Key}
Each procedure is defined by a procedure key. This key is a sequence of 24 8-bit
bytes. This is treated as a sequence of 192~bits in all cases.

The kernel itself treats the procedure keys as an opaque 192-bit value, however,
the capability system and some functions apply prefixes to these key value. For
example: capabilities that rely on procedure key values will often use a prefix
value to define a range of keys. This allows one to defined a hierarchy of
procedures, much like a directory structrue. See those capabilities for more
detail.

\subsection{Procedure Index}
Each procedure key included in the kernel is given an index, which identifies
the procedure in the procedure list.It is 1-based, that is the first value is 1,
and the value 0 is a null value.

As shown below, the maximum number of procedures is $2^{24} - 1 = 16,777,215$,
therefore the maxium value of the key index is $16,777,215$, therefore the key
index lies in the range $\left[1,1677215\right]$.

For this reason the key index is specified as 24 bytes.

\subsection{Procedure List}\label{procedure-table}
The procedure list is simply an array of storage values. The first value is the
length of this list, and the subsequent values are the procedure keys of the
list.

When a procedure is added to the kernel:

\begin{enumerate}
  \item The procedure data is added to the procedure heap (see
  Section~\ref{procedure-heap}).
  \item The procedure key is appended to the end of the array.
  \item The length value (the first value) is incremented by one.
  \item The procedure index value (in procedure metadata) is set to the new
  length value.
\end{enumerate}

When a procedure is deleted from the kernel:

\begin{enumerate}
  \item If the procedure key is the same as the Entry Procedure Key, abort and
  throw an error.
  \item If the procedure key does not exist in the list (i.e. when looking on
  the procedure heap no procedure index is associated with it), abort and throw
  an error.
  \item The length value is decremented by one.
  \item If the procedure being deleted is not the last in the list (i.e. it's
  procedure index does not equal the length of the procedure list), the last in
  the list is copied to overwrite the key being deleted. This also accounts for
  the case of an empty list.
\end{enumerate}

It is important to note that none of these steps consider deletion (zeroing) of
data. This is optional for efficiency and can be performed any time after the
length value has been decremented.

The procedure table is stored under the prefix \texttt{0xff ff ff ff 01}. The
maximum number of keys held under this configuration is $2^{24} - 1 =
16,777,215$ (1 is subtracted to account for the length value occupying a single
space). This is one less than the total number of procedures that can be held by
the kernel (which is $2^{24} = 16,777,216$).

\begin{verbatim}
  0xff ff ff ff 01 + 24 more bytes + 3 zero bytes
\end{verbatim}

\begin{table}[H]
  \caption{Procedure table.}
  \centering{}%
  \begin{tabularx}{\textwidth}{l|X}
    \hline
    Storage Location & Description\\
    \hline
    \hline
    \texttt{0xff ff ff ff 01}     & length/number of procedures
      \\
    \texttt{0xff ff ff ff 01} + keyIndex & procedure key/name: 24 bytes
      \\
    ~~~$\vdots$ & capabilities in series \\
    \texttt{0xff ff ff ff 01} + n & procedure key/name for procedure $n$
      \\
    \hline
  \end{tabularx}
\end{table}

\subsection{Procedure Heap}\label{procedure-heap}

Procedure data is stored using the key \texttt{0xff ff ff ff 00}. This is
combined with the procedure key to produce the following:

\begin{verbatim}
  0xff ff ff ff 00 + 24 bytes procedure key + 3 bytes data offset
\end{verbatim}

This leaves 3 bytes at the end of the storage location key. If the first of
these bytes is 0x00, the storage key refers to various metadata of the
procedure, including the location of its contract. If the first byte is any
other value, then the storage key corresponds to the list of capabilities of
that type. For example, the type of the write procedure is 7, so if the first of
the last 3 bytes is 0x07, then that storage key refers to a capability of the
write type.

The second value is the index into the capability list. If the second value is
0x00, this indicates the number of capabilities in this list. Therefore the
capabilities start at 0x01. For example, if the third last and second last
values are 0x0703, this refers to the third write capability (capability at
index 0x02) held by this procedure.

The very last value is an offset into the capability data. The meaning of this
is different for each capability as they each have different formats. The format
of each capability is specified in Section~\ref{syscalls-and-caps}.

\includegraphics[width=\textwidth]{media/ProcedureKeyLocation.pdf}

The following table outlines what data is found at each of the different
locations specified by the \emph{Data Offset}.

\begin{table}[H]
  \caption{Procedure data.}
  \centering{}%
  \begin{tabularx}{\textwidth}{c | c | c | c | X }
    \hline
    \multicolumn{4}{c|}{Data Offset} & Description \\
    \hline
    \hline
    \texttt{0x} & \texttt{00} & \texttt{00} & \texttt{00} & Address: 20 bytes,
      aligned right in the 32 bytes \\
    \texttt{0x} & \texttt{00} & \texttt{00} & \texttt{01} & Procedure Index: 24 bytes,
      aligned right in the 32 bytes \\
    \texttt{0x} & \texttt{$ty$} & \texttt{00} & \texttt{00} & The number of
      capabilities of type \texttt{0x$ty$}. \\
    \texttt{0x} & \texttt{$ty$} & \texttt{$in$} & \texttt{$of$} & Capability of
      type \texttt{0x$ty$}, with index \texttt{0x$in-1$}, and offset
      \texttt{0x$of$} into that capability. \\
    \hline
  \end{tabularx}
\end{table}

\section{System Calls and Capabilities}\label{syscalls-and-caps}
All system calls are transactions to the kernel. The transaction data is defined
as follows:

\begin{table}[H]
  \caption{System call structure.}
  \centering{}%
  \begin{tabularx}{\textwidth}{c|X}
    \hline
    Byte Offset & Description\\
    \hline
    \hline
    0x0 & 1 byte, system call value \\
    0x8 & The system call data as defined by each system call \\
    ~~~$\vdots$ &  \\
    \hline
  \end{tabularx}
\end{table}

If the transaction data is longer that specified by the system call format, the
additional data is simply ignored. \textbf{TODO:} We should consider throwing an
error on this.

System call numbers match capability numbers.

\begin{table}[H]
  \caption{System call and capability numbers.\label{syscall-numbers}}
  \centering{}%
  \begin{tabularx}{\textwidth}{c|X}
    \hline
    Type Value & Description\\
    \hline
    \hline
    0x0 & Null capability / noop \\
    % 0x1 & The system call data as defined by each system call \\
    % 0x2 & The system call data as defined by each system call \\
    0x3 & Call Procedure \\
    0x4 & Register Procedure \\
    0x5 & Delete Procedure \\
    0x6 & Set Entry Procedure \\
    0x7 & Write \\
    0x8 & Log \\
    0x9 & Gas Send \\
    \hline
  \end{tabularx}
\end{table}

\subsection{Executing a System Call}
The running code will issue a transaction to the kernel by perfoming a
DELEGATECALL to the kernel with the transactions message data as both above and
in the definitons of each capability. The system call will be processed as
follows:

\begin{enumerate}
  \item The kernel will receive the transaction as defined by the EVM.
  \item The kernel will read the first byte of the transaction data as an
  unsigned 8-bit integer. This value is $s$.
  \item If $s$ is not one of the values listed in Table~\ref{syscall-numbers}
  then the kernel will revert the current transaction (i.e. the DELEGATECALL)
  and return the error code \texttt{SYSCALL\_NOEXIST} (error codes are defined
  in Section~\ref{errors}).
  \item If $s$ is 0, it will do nothing and return successfully.
  \item Otherwise the control will be passed to the code in the kernel
  corresponding to the $s$. For example if the $s = 0x4$, then the code for
  executing a Register Procedure syscall will be called.
  \item This code will execute and return whatever value it deems appropriate.
\end{enumerate}

\subsection{Return and Error Codes}\label{errors}
This section defines return and error codes for interacting with the kernel.
These values are returned by system calls \emph{only} in the event of a failure
of the kernel, in which case the DELEGATECALL will have been reverted. As a
result, these error codes are provided only when the kernel reverts, if the
DELEGATECALL returns normally, whatever is returned is determined by the
procedure being called. That is, the DELEGATECALL should leave a 0 on the stack.

The error code is always 8-bits, and may followed by additional data.

Therefore:

\begin{itemize}
  \item If 1 is left on the stack after the DELEGATECALL of a system call, then
  the system call succeeded, and whatever data was returned was returned by
  system call.
  \item If 0 is left on the stack after the DELEGATECALL of a system call, then
  the system call failed, and whatever data was returned is defined in the table
  below.
\end{itemize}

\begin{table}[H]
  \caption{Return and error codes.}
  \centering{}%
  \begin{tabularx}{\textwidth}{l|c|X|X}
    \hline
    Type & Value & Additional Data & Description \\
    \hline
    \hline
    SYSCALL\_BADCAP  & 33 & None & Capability insufficient. \\
    SYSCALL\_NOGAS   & 44 & None & Procedure execution ran out of gas. \\
    SYSCALL\_REVERT  & 55 & Returned data from procedure & The called
      procedure reverted. \\
    SYSCALL\_FAIL    & 66 & System call specific error data & The system call
      failed for reasons specific to the system call. \\
    SYSCALL\_NOEXIST & 111 & None & The syscall integer values is not a valid
      syscall as defined in Table \ref{syscall-numbers}. \\
    \hline
  \end{tabularx}
\end{table}

\subsection{Capability Subsets}\label{cap-subsets}

Capability $B$ is a subset\footnote{$A$ subset, not a \emph{strict} or
\emph{proper} subset.} of Capability $A$, if everything that can be done using
Capability $B$ can be done using Capabilty $A$. The following rules apply:

\begin{itemize}
  \item $A$ is always a subset of itself. $A \subseteq A$
  \item Transitivity: $A \subseteq B \land B \subseteq C \implies A
  \subseteq C$
\end{itemize}

As the nature of each capability is different, the way subsets are defined is
different for each one. The following sections will define how subsets are
defined for each capability.

\subsection{Call System Call}

\subsubsection{System Call Format}
Type Value: 3

Data: Note that here the offset is from after system call type.

Note that here we could pack the data much more densely, but we don't in order
to err on the side of simplicity and stick as close as we can to 32-byte values.

\begin{table}[H]
  \caption{Call procedure system call format.}
  \centering{}%
  \begin{tabularx}{\textwidth}{c|X}
    \hline
    Byte Offset & Description \\
    \hline
    \hline
    \texttt{0x00} & Capability index; 32 bytes \\
    \texttt{0x20} & Prefix size, 1 bytes \\
    \texttt{0x21} & 8 unused bytes \\
    \texttt{0x28} & Procedure key, 24 bytes \\
    \hline
    End &  \\
    \hline
  \end{tabularx}
\end{table}

\subsubsection{Capability Format}
The capability format for the Call Procedure system call defines the a range of
procedure keys what the capability allows one to call. This is defined as a base
procedure key $b$ and a prefix $s$. Given this capability, a procedure may call
any procedure where the first $s$ bits of the key of that procedure key are the
same as the first $s$ bits of procedure key $b$.

The values of this capability are packed into a single 32-byte value.

\begin{table}[H]
  \caption{Call procedure capability format.}
  \centering{}%
  \begin{tabularx}{\textwidth}{c|c|X}
    \hline
    Key Offset & Byte Offset & Description \\
    \hline
    \hline
    \texttt{0x00} & \texttt{0x00} & The prefix, which is in the interval [0,24]. \\
    \texttt{0x00} & \texttt{0x01 - 0x07} & Unsed and undefined. \\
    \texttt{0x00} & \texttt{0x08 - 0x20} & The 24 bytes of the base procedure key. \\
    \hline
    End &   \\
    \hline
  \end{tabularx}
\end{table}

\includegraphics[width=\textwidth]{media/ProcedureKeyBytes.pdf}

\subsubsection{Capability Subsets}
As defined in Section~\ref{cap-subsets}, each capability has subsets and
supersets. For the call capability a subset is capability which possesses the
ability to call no more procedures than the original.

The prefix size of a capability $C$ is given as $C_s$ and the base procedure key
$C_b$. Given a call procedure capability $A$ and a proposed subset $B$:
\begin{itemize}
  \item If the prefix size of $B$ is equal to or greater than $A$ (that is: $B_s
  \geq A_s$), and
  \item If the first $B_s$ bits of $B_b$ is equal to the first $A_s$ bits of
  $A_b$.
\end{itemize}

\subsubsection{Error Codes}
If the Call Procedure system call fails for any reason other than those
general system call failure conditions, the system call will return
\texttt{SYSCALL\_FAIL} (as definted in Section~\ref{errors}) followed by one of
the following codes.

\begin{table}[H]
  \caption{Return and error codes.}
  \centering{}%
  \begin{tabularx}{\textwidth}{c|c|l|X}
    \hline
    Type & Value & Additional Data & Description \\
    \hline
    \hline
    CALL\_NOPROC  & 33 & None & The procedure key specified does not exist. \\
    \hline
  \end{tabularx}
\end{table}

\marginpar[Non-Normative]{\textbf{Non-Normative}}For example, if the Call
Procedure system call was executed with a procedure key that does not exist then
the DELEGATECALL will leave a 0 on the stack and return two bytes
(\texttt{0x6633}).

\subsection{Register Procedure System Call}

\subsubsection{System Call Format}
Type Value: 4

Data: min(96 bytes) max(96 bytes)

\begin{table}[H]
  \caption{Register procedure system call structure.}
  \centering{}%
  \begin{tabularx}{\textwidth}{c|X}
    \hline
    Byte Offset & Description\\
    \hline
    \hline
    \texttt{0x00} & Capability index; 32 bytes \\
    \texttt{0x20} & Procedure key; 24 bytes in 32 bytes \\
    \texttt{0x40} & Procedure address; 20 bytes in 32 bytes \\
    ~~~$\vdots$ & Capabilities; A series of 32 byte values which correspond
    exactly to the keys as defined in Table~\ref{reg-cap-format}. The length of
    this list is not provided. \\
    \hline
    End &  \\
    \hline
  \end{tabularx}
\end{table}

\begin{table}[H]
  \caption{Capability data as sent in the Register Procedure system call.\label{reg-cap-format}}
  \centering{}%
  \begin{tabularx}{\textwidth}{c|X}
    \hline
    Key Offset & Description \\
    \hline
    \hline
    \texttt{0x00}       & CapSize: 2 bytes, aligned right in the 32 bytes, the given
      name \\
    \texttt{0x01}       & CapType: 1 byte, aligned right in the 32 bytes \\
    \texttt{0x02}       & key value \#1: 32 bytes \\
    ~~~$\vdots$ & capabilities in series \\
    CapSize & final key value \\
    \hline
  \end{tabularx}
\end{table}

\subsubsection{Capability Format}
The capability format for the Register Procedure system call defines the a range
of procedure keys what the capability allows one to call. This is defined as a
base procedure key $b$ and a prefix $s$. Given this capability, a procedure may
call any procedure where the first $s$ bits of the key of that procedure key are
the same as the first $s$ bits of procedure key $b$.

\textbf{NB:} Register procedure also relies on all of the other capabilities
possessed by a procedure to determine what capabilities the new procedure will
have.

\begin{table}[H]
  \caption{Register procedure capability format.}
  \centering{}%
  \begin{tabularx}{\textwidth}{c|c|X}
    \hline
    Key Offset & Byte Offset & Description \\
    \hline
    \hline
    \texttt{0x00} & \texttt{0x00} & The prefix, which is in the interval [0,24]. \\
    \texttt{0x00} & \texttt{0x01 - 0x07} & Unsed and undefined. \\
    \texttt{0x00} & \texttt{0x08 - 0x20} & The 24 bytes of the base procedure key. \\
    \hline
    End &   \\
    \hline
  \end{tabularx}
\end{table}

\includegraphics[width=\textwidth]{media/ProcedureKeyBytes.pdf}

\subsubsection{Capability Subsets}
As defined in Section~\ref{cap-subsets}, each capability has subsets and
supersets. For the register procedure capability a subset is capability which
possesses the ability to register no more procedures than the original.

The prefix size of a capability $C$ is given as $C_s$ and the base procedure key
$C_b$. Given a register procedure capability $A$ and a proposed subset $B$:
\begin{itemize}
  \item If the prefix size of $B$ is equal to or greater than $A$ (that is: $B_s
  \geq A_s$), and
  \item If the first $B_s$ bits of $B_b$ is equal to the first $A_s$ bits of
  $A_b$.
\end{itemize}

\subsubsection{Function}
This registers a new procedure which already exists as a contract at a certain
address, giving it a name an a list of capabilities. It does a number of things:

\begin{itemize}
  \item Validate the code at the given address to show that it complies with the
  requirements of procedure code as described in
  Section~\ref{bytecode-validation}.
  \item Add the procedure name to the procedure list.
  \item Store the specified capabilities with the procedure, but only if each of
  the capabilities is found to be a subset of one of the capabilities of the
  procedure performing this system call.
\end{itemize}

There are two highly critical functions here upon which the safety of the kernel
relies. The first is the validation of the procedure bytecode, which is covered
in Section~\ref{bytecode-validation}, but it is also necessary to ensure that
the capabilities that are being asked to be given to the new procedure can be
provided by the current procedure. In order to satisfy this constraint it must
be shown that for every requested procedure, there exists a capability in the
capability list of the current procedure. It is currently not able to
``combine'' capabilities. That is, if the procedure has a capability of
Write(0x80,5) and Write(0x85,5) it cannot provide a capability Write(0x80,10)
even though it theoretically still able to perform the same writes.

The algorithm is as follows:

\begin{enumerate}
  \item For each of the capabilities in the list of requested capabilities,
  search though the list of the current procedures capabilities until a superset
  of the requested capability is found.
  \item If for any of the requested capabilities a superset is not found, abort
  the entire process.
  \item If for all of the requested capabilities a superset is found, register
  the capability.
\end{enumerate}

\subsubsection{Error Codes}
If the Register Procedure system call fails for any reason other than those
general system call failure conditions, the system call will return
\texttt{SYSCALL\_FAIL} (as definted in Section~\ref{errors}) followed by one of
the following codes.

\begin{table}[H]
  \caption{Return and error codes.}
  \centering{}%
  \begin{tabularx}{\textwidth}{c|c|l|X}
    \hline
    Type & Value & Additional Data & Description \\
    \hline
    \hline
    REG\_TOOMANYCAPS  & 77 & None & Too many caps were provided. \\
    \hline
  \end{tabularx}
\end{table}

\marginpar[Non-Normative]{\textbf{Non-Normative}}For example, if the Register
Procedure system call was executed but provides too many capabilities then the
DELEGATECALL will leave a 0 on the stack and return two bytes (\texttt{0x6677}).

\subsection{Delete Procedure System Call}

\subsubsection{System Call Format}
Type Value: 5

Data: min(64 bytes) max(64 bytes)

\begin{table}[H]
  \caption{Delete procedure system call.}
  \centering{}%
  \begin{tabularx}{\textwidth}{c|X}
    \hline
    Byte Offset & Description\\
    \hline
    \hline
    \texttt{0x00} & Capability index; 32 bytes \\
    \texttt{0x20} & Procedure key; 24 bytes in 32 bytes \\
    \hline
    End &  \\
    \hline
  \end{tabularx}
\end{table}

\subsubsection{Capability Format}
The capability format for the Delete Procedure system call defines the a range
of procedure keys what the capability allows one to delete. This is defined as a
base procedure key $b$ and a prefix $s$. Given this capability, a procedure may
delete any procedure where the first $s$ bits of the key of that procedure key are
the same as the first $s$ bits of procedure key $b$.

\begin{table}[H]
  \caption{Delete procedure capability format.}
  \centering{}%
  \begin{tabularx}{\textwidth}{c|c|X}
    \hline
    Key Offset & Byte Offset & Description \\
    \hline
    \hline
    \texttt{0x00} & \texttt{0x00} & The prefix, which is in the interval [0,24]. \\
    \texttt{0x00} & \texttt{0x01 - 0x07} & Unsed and undefined. \\
    \texttt{0x00} & \texttt{0x08 - 0x20} & The 24 bytes of the base procedure key. \\
    \hline
    End &   \\
    \hline
  \end{tabularx}
\end{table}

\includegraphics[width=\textwidth]{media/ProcedureKeyBytes.pdf}

\subsubsection{Capability Subsets}
As defined in Section~\ref{cap-subsets}, each capability has subsets and
supersets. For the delete procedure capability a subset is capability which
possesses the ability to delete no more procedures than the original.

The prefix size of a capability $C$ is given as $C_s$ and the base procedure key
$C_b$. Given a delete procedure capability $A$ and a proposed subset $B$:
\begin{itemize}
  \item If the prefix size of $B$ is equal to or greater than $A$ (that is: $B_s
  \geq A_s$), and
  \item If the first $B_s$ bits of $B_b$ is equal to the first $A_s$ bits of
  $A_b$.
\end{itemize}

\subsection{Set Entry Procedure System Call}
The value of the Entry Procedure is 24-byte procedure key which identified which
procedure should be first called upon receiving a transaction. This is currently
store at the storage location:

\texttt{0x00 00 00 00 {\color{blue} 04} 00 $\cdots$ 00 00 00 00}

\subsubsection{Error Codes}
If the Delete Procedure system call fails for any reason other than those
general system call failure conditions, the system call will return
\texttt{SYSCALL\_FAIL} (as definted in Section~\ref{errors}) followed by one of
the following codes.

\begin{table}[H]
  \caption{Return and error codes.}
  \centering{}%
  \begin{tabularx}{\textwidth}{c|c|l|X}
    \hline
    Type & Value & Additional Data & Description \\
    \hline
    \hline
    DEL\_NOPROC  & 33 & None & The procedure key specified does not exit. \\
    \hline
  \end{tabularx}
\end{table}

\marginpar[Non-Normative]{\textbf{Non-Normative}}For example, if the Delete
Procedure system call was executed with a procedure key that does not exist then
the DELEGATECALL will leave a 0 on the stack and return two bytes
(\texttt{0x6633}).

\subsubsection{System Call Format}

Type Value: 6

Data: min(64 bytes) max(64 bytes)

\begin{table}[H]
  \caption{Set entry procedure system call format.}
  \centering{}%
  \begin{tabularx}{\textwidth}{c|X}
    \hline
    Byte Offset & Description\\
    \hline
    \hline
    \texttt{0x0} & Capability index; 32 bytes \\
    \texttt{0x20} & Procedure key; 24 bytes in 32 bytes \\
    \hline
    End &  \\
    \hline
  \end{tabularx}
\end{table}

\subsubsection{Capability Format}
This capability has no associated data and is therefore of size zero.

\subsubsection{Capability Subsets}
As defined in Section~\ref{cap-subsets}, each capability has subsets and
supersets. As the Set Entry Procedure capability is a trivial value, there is
only one value for Set Entry Procedure capabilities and they are all equal.

Given a Set Entry Procedure capability $A$ and any other Set Entry Procedure
capability $B$, not only is $B \subseteq A$ but also $B = A$.

\subsubsection{Function}
The only possible state change is the "entryProcedure" value. No other state
changes should occur.

\subsubsection{Error Codes}
If the Set Entry Procedure system call fails for any reason other than those
general system call failure conditions, the system call will return
\texttt{SYSCALL\_FAIL} (as definted in Section~\ref{errors}) followed by one of
the following codes.

\begin{table}[H]
  \caption{Return and error codes.}
  \centering{}%
  \begin{tabularx}{\textwidth}{c|c|l|X}
    \hline
    Type & Value & Additional Data & Description \\
    \hline
    \hline
    SETENT\_NOPROC  & 33 & None & The procedure key specified does not exit. \\
    \hline
  \end{tabularx}
\end{table}

\marginpar[Non-Normative]{\textbf{Non-Normative}}For example, if the Set Entry
Procedure system call was executed with a procedure key that does not exist then
the DELEGATECALL will leave a 0 on the stack and return two bytes
(\texttt{0x6633}).

\subsection{Write System Call}
This system call will write a single 32-byte value under a single 32-byte key in
the storage of the kernel instance.

\subsubsection{System Call Format}
Type Value: 7

Data:

\begin{table}[H]
  \caption{System call structure.}
  \centering{}%
  \begin{tabularx}{\textwidth}{c|X}
    \hline
    Byte Offset & Description\\
    \hline
    \hline
    \texttt{0x00} & Capability index; 32 bytes \\
    \texttt{0x20} & Write address; 32 bytes \\
    \texttt{0x40} & Write values; 32 bytes \\
    \hline
    End &  \\
    \hline
  \end{tabularx}
\end{table}

\subsubsection{Capability Format}
The write capability includes 2 values: the first is the base address where we
can write to storage. The second is the number of \emph{additional} addresses we
can write to. For example, if the first value is 0x8000, and the second value is
0, we can write only to location 0x8000. If the second value was 5, we could
write to 0x8000, 0x8000+1, 0x8000+2, 0x8000+3, 0x8000+4, and 0x8000+5.

\begin{table}[H]
  \caption{Write capability format.}
  \centering{}%
  \begin{tabularx}{\textwidth}{c|c|X}
    \hline
    Key Offset & Value & Description\\
    \hline
    \hline
    \texttt{0x00} & $a$ & The base storage address\\
    \texttt{0x01} & $n$ & The number of additional keys we can write to\\
    \hline
    End &  \\
    \hline
  \end{tabularx}
\end{table}

\subsubsection{Capability Subsets}
As defined in Section~\ref{cap-subsets}, each capability has subsets and
supersets.

Given a Write capability $A$ and a proposed subset $B$, where $C_a$ is the base
address of capability $C$ and $C_n$ is the number of additional keys writable
with capability $C$, $B \subseteq A \iff \left(B_a \geq A_a\right) \land
\left(B_a + B_n \leq A_a + A_n\right)$.

Or more verbosely, $B$ is only a subset of $A$ if and only if:
\begin{itemize}
  \item The lowest writable address (which is the base address) of $B$ is
  greater than or equal to the lowest writable address of $A$, and
  \item The highest writable address (which is base address plus the number of
  additional keys) of $B$ is less than or equal to the highest writable address
  of $A$.
\end{itemize}

\subsubsection{Error Codes}
There are no specific error conditions for SSTORE, therefore only general system
call errors are applicable.

\subsection{Log System Call}

\subsubsection{System Call Format}
Type Value: 8

Data: min(96 bytes) max(224 bytes)

\begin{table}[H]
  \caption{System call structure.}
  \centering{}%
  \begin{tabularx}{\textwidth}{l|X}
    \hline
    Byte Offset & Description \\
    \hline
    \hline
    \texttt{0x00} & Capability index; 32 bytes \\
    \texttt{0x20} & Number of topics (nTopics) \\
    ... & Potentially topic value \#1; 32 bytes \\
    ... & Potentially topic value \#2; 32 bytes \\
    ... & Potentially topic value \#3; 32 bytes \\
    ... & Potentially topic value \#4; 32 bytes \\
    \texttt{0x40}+(nTopics $\times$ 0x20) & Log value 32 bytes \\
    \hline
    End &  \\
    \hline
  \end{tabularx}
\end{table}

\subsubsection{Capability Format}
The Write capability includes between 0 and 4 values. Each value forces the use
of a particular value. For example: if the capability has 1 value, then that
means that the first topic in the log call must equal that. If it has 2 values
then the first log topic must be the first of those values and the second topic
must be the second of those values and so on. If there are no topics listed then
there are no restrictions on what can be logged.

This capability is 5 values. If the number of enforced topics is less than 4,
then the unused values are undefined.

\begin{table}[H]
  \caption{Log capability format.}
  \centering{}%
  \begin{tabularx}{\textwidth}{c|c|X}
    \hline
    Key Offset & Value & Description\\
    \hline
    \hline
    \texttt{0x00} & [0x00, 0x4] & Number of enforced topics; 32 bytes \\
    \texttt{0x01} & $t_1$ & Potentially an enforced value for the first topic
      \\
    \texttt{0x02} & $t_2$ & Potentially an enforced value for the second topic
      \\
    \texttt{0x03} & $t_3$ & Potentially an enforced value for the third topic
      \\
    \texttt{0x04} & $t_4$ & Potentially an enforced value for the fourth topic
      \\
    \hline
    End &  \\
    \hline
  \end{tabularx}
\end{table}

\subsubsection{Capability Subsets}
As defined in Section~\ref{cap-subsets}, each capability has subsets and
supersets.

Given a Log capability $A$ and a proposed subset $B$, $B \subseteq A$ if and
only if for each log topic of $A$ the topic is either undefined or equal to that
of $B$.

If $C_{t_i}$, where $i$ is $1$, $2$, $3$, or $4$, is the log topic $i$ of
capability $C$, and $C_n$ is the number of defined topics:

$$ \forall i \in \left\{1,2,\dots ,A_n\right\}. B_{t_i} = A_{t_i} $$

\subsubsection{Error Codes}
There are no specific error conditions for SSTORE, therefore only general system
call errors are applicable.

\subsection{Value Send System Call}
This system call sends value (Ether) to another account.

\subsubsection{System Call Format}
Type Value: 9

Data:

\begin{table}[H]
  \caption{System call structure.}
  \centering{}%
  \begin{tabularx}{\textwidth}{c|X}
    \hline
    Byte Offset & Description\\
    \hline
    \hline
    \texttt{0x00}& Capability index; 32 bytes \\
    \texttt{0x20} & Account address; 20-bytes right aligned in 32 bytes \\
    \texttt{0x40} & Value amount; 32 bytes \\
    \hline
    End &  \\
    \hline
  \end{tabularx}
\end{table}

\subsubsection{Capability Format}
This value contains no data other than its presence, therfore has no format.

\subsubsection{Capability Subsets}
As defined in Section~\ref{cap-subsets}, each capability has subsets and
supersets. As the Value Send capability is a trivial value, there is only one
value for Value Send capabilities and they are all equal.

Given a Value Send capability $A$ and any other Value Send capability $B$, not
only is $B \subseteq A$ but also $B = A$.

\section{Procedure Bytecode Validation}\label{bytecode-validation}
In order to be added to the procedure list of a kernel, the bytecode of a contract must be validated by the kernel and must conform to a strict set of requirements. These requirements are as follows:

\begin{itemize}
  \item All the opcodes must be on the whitelist as defined in Section~\ref{opcode-whitelist}, unless part
  of a system call.
  \item The first opcodes of the contract must form a valid execution guard as
  defined in Section~\ref{execution-guard}.
\end{itemize}


\subsection{Whitelist}\label{opcode-whitelist}
Generally the whitelist consists of all of the non-state-changing opcodes. This
\emph{must} be implemented as a whitelist and not a blacklist, as this should
continue to work in the case that new state-changing opcodes are added to the
VM.

The table below contains the ranges of opcodes that are allowed.

\begin{table}[H]
  \caption{Opcode whitelist.}
  \centering{}%
  \begin{tabularx}{\textwidth}{c | X }
    \hline
    Range (Inclusive) & Description \\
    \hline
    \hline
    \texttt{0x00} - \texttt{0x0b} & Stop and Arithmetic \\
    \texttt{0x10} - \texttt{0x1a} & Comparison \& Bitwise Logic Operations \\
    \texttt{0x20} & SHA3 \\
    \texttt{0x30} - \texttt{0x3e} & Environmental Information \\
    \texttt{0x40} - \texttt{0x45} & Block Information \\
    \texttt{0x50} - \texttt{0x53} & Stack, Memory, Storage and Flow Operation - Part 1\\
    \texttt{0x54} & SLOAD \\
    \texttt{0x56} - \texttt{0x5b} & Stack, Memory, Storage and Flow Operation - Part 2 \\
    \texttt{0x80} - \texttt{0x8f} & Duplication Operations \\
    \texttt{0x90} - \texttt{0x9f} & Exchange Operations \\
    \texttt{0xf3} & Return \\
    \texttt{0x60} - \texttt{0x7f} & Push \\
    \texttt{0xfa} & STATICCALL \\
    \texttt{0xfd} & REVERT \\
    \texttt{0xfe} & INVALID \\
    \hline
  \end{tabularx}
\end{table}

\subsection{Execution Guard}\label{execution-guard}
An execuation guard \emph{must} start at position \texttt{0x00} in the bytecode.
It is defined as follows:

\definecolor{mygreen}{rgb}{0,0.6,0}
\begin{minipage}{\linewidth}
\small
\begin{lstlisting}[language=c,commentstyle=\color{mygreen},basicstyle=\ttfamily,
  identifierstyle=\color{blue},
  caption=Sequence of steps which constitutes an execution guard.]
0x00: PUSH 0xffffffff0200...00 // Push kernel addr. location
0x21: SLOAD     // Load value to the stack
0x22: PUSH 0x2a // Load value to the stack
0x24: JUMPI     // Jump if kernel address non-zero
0x25: PUSH1 0x00 // Revert data size
0x27: PUSH1 0x00 // Revert data location
0x29: REVERT    // Revert because we are not a kernel
0x2a: JUMPDEST  // Destination to jump over
\end{lstlisting}
\end{minipage}

\subsection{System Call}
A system call is in the form:

\definecolor{mygreen}{rgb}{0,0.6,0}
\begin{minipage}{\linewidth}
\small
\begin{lstlisting}[language=c,commentstyle=\color{mygreen},basicstyle=\ttfamily,
  identifierstyle=\color{blue},
  caption=Sequence of steps to perform a system call.]
CALLER       // Get Caller
GAS          // Put all the available gas on the stack
DELEGATECALL // Delegate Call to Caller
\end{lstlisting}
\end{minipage}

\end{document}
